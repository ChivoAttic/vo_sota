\subsection{Summary of Services and Tools}
%%%%%%%%%%%%%%%%%%%%%%%%%%%%%%%%%%%%%%%%%%%%%%%%%%%%%%%%%%%%%%%%%%%%%%%%%%%%%%%%%
%%%%%%%%%%%%%%%%%%%%%%%%%%%%%%%%%$Data Access%%%%%%%%%%%%%%%%%%%%%%%%%%%%%%%%%%%%
%%%%%%%%%%%%%%%%%%%%%%%%%%%%%%%%%%%%%%%%%%%%%%%%%%%%%%%%%%%%%%%%%%%%%%%%%%%%%%%%%
\subsection{Data Access and Local Scientific processing tools}
\begin{table*}[h!t]
	\centering
	\begin{tabular}{|l|l|p{12.5cm}|}
	\hline
	\textbf{VO} & \textbf{Project} & \textbf{Description}\\
	\hline
	CVO 	& Data Sharing (VOSpace 2.0) &  A service that allows to share files and collaborate with team members \\
			& Table Access Protocol (TAP-1.0) &	A model that implements a standard view for \textbf{Table Access Protocol (TAP-1.0)}. \\
			& Observation Model Core Components & A model that implements a standard view for \textbf{Table Access Protocol (TAP-1.0)}. \\
			& Simple Image Access (SIA-1.0) & a SIA-1.0 compliant query service for easy access to calibrated images from most our data collections. \\
	\hline
	VAO		& Data Discovery Tool & A web tool that allows to find datasets from astronomical collections known to the VO like the Hubble Space Telescope 
									(HST), the Chandra X-ray Observatory, the Spitzer Space Telescope, among other.\\
			& Time Series Search Tool & A web tool that allows to access the time series data sets at the Harvard Time Series Center (TSC), the NASA 
									Exoplanet Archive and the Catalina Real-Time Transient Survey, and analize them with the periodogram application 
									of the NASA Exoplanet Archive.\\
			& Iris: SED Analysis Tool & A downloadable application for the finding, plotting and fitting the 
			Spectral Energies Distributions (SEDs). \\
	\hline
	BRAVO	& BRAVO@LNA & Making of a virtual observatory dedicated to Southern Astrophysical Research Telescope (SOAR) data from Brazilian astronomers.  \\
			& BRAVO@UFSC & Researching of the of the power spectral synthesis as a mean to estimate the physical properties of the galaxies.\\
			& CYCLOPS & A software that models the optical emission from AM Her systems including the four Stokes parameters.\\
	\hline
	ChiVO	& Conceptual Design of a VO for ALMA & A degree thesis that through the researching and analysis of 
									query languages, formats and the semantic of OVA (in its Spanish acronym), and the definition of it intends to make a conceptual design for the ALMA 
									observatory.\\
			& Search for Astronomical Patterns & A degree thesis that through 
									the researching and analysis o search and recognition methods of 
									patterns intends to minimize the search space with minimal impact on the 
									results. \\
	\hline								
	HVO		& Spectrum Service for VO & A proposal that intends to add several features and make two substantial improvements to the web services that 
									contains spectra of galaxies and the other astronomical objects.\\
			& Synthetic Spectrum Service & A proposal that intends to serve, as a web service, the ready made spectra for the users.\\
			& Debrecen Photoheliographic Data & A sunspot catalogue with the heliographic positions and the areas of sunspots. A continuation of Greenwich 
									Photoheliographic Results (GPR) that had been discontinued on 1976.\\
	\hline								
	AstroGrid& Topcat & An interactive graphical viewer and editor for tabular data for formats like the Flexible Image Transport System (FITS) 
									and VOTable. \\
			& VODesktop & an analysis tools wich allows limit the choice of resources through specific data saving.\\
	\hline		
	ESA-VO	& DALToolKit & A downloadable software based on JAVA that allows to publish in the VO following the Data Access Layer (DAL) protocol. It 
									converts incoming standard DAL requests into database specific SQL queires, then serializes the database result into VOTable 
									responses. \\
			& IVOA Resource Registry & the official EURO-VO resource registry under the name of EURO-VO Full Harvestable VO Resource Registry that \\
			& VOSpec & A multi-wavelength spectral analysis tool with access to atomic and molecular databases, spectra and theorical models registered 
									in the VO. It can be downloadable and accessed through Internet.\\
	\hline								
	GAVO	& GAVO Data Center & A growing collection of data and services provided on behalf of third parties. Some of the GAVO services are also 
									available on \url{http://dc.zah.uni-heidelberg.de/}\\
			& MPA Simulations access & A web service for querying the results of the Millennium simulation using SQL.\\
			& MultiDark Database & A service wich gives access to data from MultiDark and Bolshoi simulations using SQL queries.  It based on the Millennium 
									Web Application. \\
			& RAVE archive search & An access to a growing archive of radial velocities for more than 400 000 stars.\\
			& TheoSSA & A service for providing spectral energy distributions based on model atmosphere calculations.\\
	\hline		
	SVO		& TESELA & a service that allows to access the catalog of blank regions. It is based on the application of the Delaunay triangulation
									 of the sky. \\
	\hline								 
	VObs.it	& SIAP & A web services that provides the public Hubble Space Telescope/Advanced Camera for Surveys (HST/ACS) Great Observatories Origins 
									Deep Survey (GOOD) data within the VIMOS\footnote{VIMOS is a \textbf{VI}sible imaging \textbf{M}ulti-\textbf{O}bject 
									\textbf{S}pectrograph, a spectrograph for the European Southern Observatory Very Large Telescope array (ESO-VLT).}-VLT
									Deep Survey-Chandra Deep Field South (VVDS-CDFS).\\
			& SSAP & A web service that allows to access the VVDS-F02-DEEP spectra. \\
			& Cone Search & A web service that allows to query in the VVDS-CDFS catalog.  \\
			& Skynode & A web service that allows to query in the VVDS catalogs. \\
	\hline		
	China-VO& FITS Manager & A downloadable application, as a collaboration project between China-V0 and VOI, to manage FITS, VOTable, among other files, 
								hosted in personal computers \footnote{FITS Manager (FM) is downloadble from \url{http://fm.china-vo.org/app/fm.zip}.}  \\
			& VO Data Access Service & A data access framework that allows to query large volumes of astronomical resources like catalogs, spectrum 
									and images through the command line, a graphical interface or webpage.\\
			& FITS Header Archiving System & A downloadable tool that allows to view and import the FITS header files into a database table for single or 
									multiple files. It was also developed by the IBM Center, Tianjin University (TU) and e-Science application research 
									center, Computer Network Information Center (CNIC), Chinese Academy of Sciences (CAS).\\
			& Imaging Processing and analysis tool & a downloadable imaging processing and analysis tool developed in JAVA that allows to visualize sky 
									images and access related data from the Beijing Astronomical Data Center (BADC). \\
			& VOTable2XHTML & A XSLT stylesheet that can be used to transform VOTable file into XHTML file.\\
			& SkyMouse & A search engine that allows to access astronomical services like web service and CGI service. \\
	\hline
	ArVO	& Sci1 & ``Search for new interesting objects of definite types by
low-dispersion template spectra'': ``modeling of spectra [...] [for a] QSOs,
							Seyfert galaxies, white dwarfs, [...] late-type stars (K-M, S, carbon)'' \\
			& Sci2 & ``Optical identifications of new gamma, X-ray, IR and radio
sources'': using the Byurakan 2.6 [m] telescope, ``the first test resulted in 145 
							objects found, 81 being known QSOs/Sys, and 64 new candidates (including 23 NVSS and FIRST radio sources)''.\\
			& Sci3 & ``Identification of the newly found IR sources from Spitzer
Space Telescope (SST)'': ``73 unidentified sources in the Bootes region have been 
							found and clasified on the DFBS plates''\footnote{Mickaelian, A. (2006, August). Science projects with the Armenian Virtual 
							Observatory (Arvo). Karel A.  van der Hucht (Ed.), \textit{Highlights of Astronomy} (p. 529). Vol. 14. Prague: Cambridge 
							University Press.}.\\
	\hline
%	ESA-VO	& Science Activities in the VO & development at the European Space Astronomy Centre (ESAC) of research projects based on VO, tutorial that teachs to use the tools made by the Science Archives Team (SAT), among other.
	ESA-VO & Science Activities in the VO & development at the European Space
Astronomy Center (ESAC) of research projects based on VO, tutorial that teachs
to use the tools made by the Science Archives Team (SAT), among other.\\
\hline
%	\hline
JVO   & JVO portal service & A site as portal to various kind of astronomical
resources from the Subaru Telescope, Sloan Digital Sky and ALMA, among other. \\
	\hline								
	VOI		& VOIPortal & An entry to all VOI web services. Can be browse the data downloading or through VOIMosaic and PyMorph web applications.\\
			& Mosaic Service & A software that allows to make mosaic, with SWarp\footnote{The SWarp tool is available from \url{http://www.astromatic.
									net/software/swarp}} and SExtractor\footnote{The SExtractor tool is available from \url{http://www.astromatic.
									net/software/sextractor}} and STIFF, of images retrieved from SDSS\footnote{The SDSS image server is available from 
									\url{http://casjobs.sdss.org/vo/DR7SIAP/SIAP.asmx}}, 2MASS\footnote{The 2MASS image server is available from 
									\url{http://irsa.ipac.caltech.edu/applications/2MASS/IM/}} and HST\footnote{The HST image server is available from 
									\url{http://archive.stsci.edu/siap/search.php}} image servers.\\
			& PyMorph Service & A software that allows to derive morphological parameters for galaxy images. Is possible to provide to it the output FITS
									files generated by Mosaic Service.\\
			& VOPlot & A software tool developed in JAVA that allows to visualize astronomical data available in VOTable, ASCII and FITS formats.\\
			& VOMegaPlot & a software tool developed in JAVA that allows to visualize astronomical data available in VOTable format. It looks just like 
									VOPlot. There is a client-server version.\\
			& AstroStat & a software tool that allows astronomers to use both and sophisticated statical routines on large datasets uploaded in VOTable or
									ASCII format. \\
			& VOCat & a software tool that converts astronomical catalogs to MySQL databases. \\
			& VOPlatform& a software tool developed in JAVA that allows to place their frequently used VO tools and datasets with others resourcers like 
									documents, links, among other. \\
			& VOConvert & A software tool that converts ASCII to VOTable files, FITS to VOTable and VOTable to ASCII. \\
			& Android Cosmological Calculator & an Android aplication that allows to input the Hubble constant, $ \Omega_{m} $(matter), $ \Omega_{\lambda} 
									$ (vacuum) and the redshift($ z $), and returns the current age of the Universe, the co-moving radial distance and 
									volume and the angular size distance at the specified redshift, and the luminosity distance.\\
			& Android Name Resolver & An Android application that allows to input the name of celestial object and returns information of this like 
									RA/DEC values, redshift, proper motion, parallax, among other.\\
			& CSharpFITS Package & A C\# .NET port of Tom McGlynn's nom.tam.fits JAVA packages\footnote{The C\# .NET port of Tom McGlynn's nom.tam.fits 
									JAVA packages are availablre from \url{http://heasarc.gsfc.nasa.gov/docs/heasarc/fits/java/v0.9/javadoc/}}.\\
			& VOTable JAVA Streaming Writer & A software that converts data streams in non-VOTable format, like ASCII or FITS, to the VOTable format. \\
			& C++ parser for VOTable & A C++ library to access VOTable files. It has a non-streaming and streaming version.\\
			& Fits Manager & A web-based tool for viewing, creating, adding extensions and converting FITS files.\\
			& HCT Data Archive System & a web-based system that archives the observational data generated by the Himalayan Chandra Telescope (HCT), a 2 [m]
									 aperture optical-infrared telescope manufactured by the EOS Technologies Inc. and remotely operated via dedicated
									 satellite link.\\
	\hline
	\end{tabular}
	\caption{Data Access}
	\label{table:da}
\end{table*}

%\begin{itemize}
%\item \textbf{CVO}
%\item \textbf{Data Sharing (VOSpace 2.0)}:
%a service that allows to share files and collaborate with team members.

%\item \textbf{Table Access Protocol (TAP-1.0)}:
%a service that allows the access to all the data
%described by the Common Archive Observation Model (CAOM) in use at the CADC and
%tables from other projects.

%\item \textbf{Observation Model Core Components (ObsCore-1.0)}:
%a model that implements a standard view for \textbf{Table Access Protocol (TAP-1.0)}.

%\item \textbf{Simple Image Access (SIA-1.0)}:
%a SIA-1.0 compliant query service for easy access to calibrated images from most
%our data collections.

%\item \textbf{VAO}
%\item \textbf{Data Discovery Tool}:
%a web tool that allows to find datasets from astronomical collections known to
%the VO like the Hubble Space Telescope (HST), the Chandra X-ray Observatory, the
%Spitzer Space Telescope, among other.

%\item \textbf{Time Series Search Tool}:
%a web tool that allows to access the time series data sets at the Harvard Time
%Series Center (TSC), the NASA Exoplanet Archive and the Catalina Real-Time
%Transient Survey, and analize them with the periodogram application of the NASA
%Exoplanet Archive.

%\item \textbf{BRAVO}
%\item \textbf{BRAVO@LNA}:
%making of a virtual observatory dedicated to Southern Astrophysical Research
%Telescope (SOAR) data from Brazilian astronomers.  

%\item \textbf{ChiVO}
%\item \textbf{Conceptual Design of a Virtual Astronomical Observatory for ALMA}:
%a degree thesis that through the researching and analysis of query languages,
%formats and the semantic of OVA (in its Spanish acronym), and the definition of
%it intends to make a conceptual design for the ALMA observatory.

%\item \textbf{HVO}
%\item \textbf{Spectrum Service for VO}:
%a proposal that intends to add several features and make two substantial
%improvements\footnote{Does not specified what several features and the two
%substantial improvements.} to the web services that contains spectra of galaxies
%and the other astronomical objects.

%\item \textbf{Synthetic Spectrum Service}:
%a proposal that intends to serve, as a web service, the ready made spectra for
%the users.

%\item \textbf{Debrecen Photoheliographic Data (DPD)}:
%a sunspot catalogue with the heliographic positions and the areas of sunspots. A
%continuation of Greenwich Photoheliographic Results (GPR) that had been
%discontinued on 1976.

%\item \textbf{AstroGrid}
%\item \textbf{Topcat}:
%an interactive graphical viewer and editor for tabular data for formats like the
%Flexible Image Transport System (FITS) and VOTable.

%\item \textbf{VODesktop}:
%an analysis tools wich allows limit the choice of resources through specific
%data saving.

%\item \textbf{ESA-VO}
%\item \textbf{DALToolKit}:
%a downloadable software based on JAVA that allows to publish in the VO following
%the Data Access Layer (DAL) protocol. It converts incoming standard DAL requests
%into database specific SQL queires, then serializes the database result into
%VOTable responses. 

%\item \textbf{IVOA Resource Registry}:
%the official EURO-VO resource registry under the name of EURO-VO Full
%Harvestable VO Resource Registry that 

%\item \textbf{VOSpec}:
%a multi-wavelength spectral analysis tool with access to atomic and molecular
%databases, spectra and theorical models registered in the VO. It can be
%downloadable and accessed through Internet.

%\item \textbf{GAVO}
%\item \textbf{GAVO Data Center}:
%a growing collection of data and services provided on behalf of third parties.
%Some of the GAVO services are also available on
%\url{http://dc.zah.uni-heidelberg.de/}

%\item \textbf{MPA Simulations access}:
%a web service for querying the results of the Millennium simulation using SQL.

%\item \textbf{MultiDark Database}:
%a service wich gives access to data from MultiDark and Bolshoi simulations using
%SQL queries.  It based on the Millennium Web Application.

%\item \textbf{RAVE archive search}:
%an access to a growing archive of radial velocities for more than 400 000 stars.

%\item \textbf{TheoSSA}:
%a service for providing spectral energy distributions based on model atmosphere
%calculations.

%\item \textbf{SVO}
%\item \textbf{TESELA}:
%a service that allows to access the catalog of blank regions. It is based on the
%application of the Delaunay triangulation of the sky.

%\item \textbf{VObs.it}
%\item \textbf{SIAP}:
%a web services that provides the public Hubble Space Telescope/Advanced Camera
%for Surveys (HST/ACS) Great Observatories Origins Deep Survey (GOOD) data within
%the VIMOS\footnote{VIMOS is a \textbf{VI}sible imaging
%\textbf{M}ulti-\textbf{O}bject \textbf{S}pectrograph, a spectrograph for the
%European Southern Observatory Very Large Telescope array (ESO-VLT).}-VLT Deep
%Survey-Chandra Deep Field South (VVDS-CDFS).

%\item \textbf{SSAP}:
%a web service that allows to access the VVDS-F02-DEEP spectra.

%\item \textbf{CONE SEARCH}:
%a web service that allows to query in the VVDS-CDFS catalog. 

%\item \textbf{SKYNODE}:
%a web service that allows to query in the VVDS catalogs. 

%\item \textbf{China-VO}
%\item \textbf{FITS Manager (FM)}:
%a downloadable application, as a collaboration project between China-V0 and VOI,
%to manage FITS, VOTable, among other files, hosted in personal computers
%\footnote{FITS Manager (FM) is downloadble from
%\url{http://fm.china-vo.org/app/fm.zip}.} 

%\item \textbf{VO Data Access Service (VO-DAS)}:
%a data access framework that allows to query large volumes of astronomical
%resources like catalogs, spectrum and images through the command line, a
%graphical interface or webpage.

%\item \textbf{FITS Header Archiving System (FitHAS)}:
%a downloadable tool that allows to view and import the FITS header files into a
%database table for single or multiple files. It was also developed by the IBM
%Center, Tianjin University (TU) and e-Science application research center,
%Computer Network Information Center (CNIC), Chinese Academy of Sciences (CAS).

%\item \textbf{Imaging Processing and analysis tool for China\_VO (VO\_IMPAT)}:
%a downloadable imaging processing and analysis tool developed in JAVA that
%allows to visualize sky images and access related data from the Beijing
%Astronomical Data Center (BADC). 

%\item \textbf{VOTable2XHTML}
%a XSLT stylesheet that can be used to transform VOTable file into XHTML file.

%\item \textbf{SkyMouse}:
%a search engine that allows to access astronomical services like web service and
%CGI service.

%\item \textbf{JVO}
%\item \textbf{JVO portal service}:
%a site as portal to various kind of astronomical resources from the Subaru
%Telescope, Sloan Digital Sky and ALMA, among other.

%\item \textbf{VOI}
%\item \textbf{VOIPortal}:
%an entry to all VOI web services. Can be browse the data downloading or through
%VOIMosaic and PyMorph web applications.

%\item \textbf{Mosaic Service}:
%a software that allows to make mosaic, with SWarp\footnote{The SWarp tool is
%available from \url{http://www.astromatic.net/software/swarp}} and
%SExtractor\footnote{The SExtractor tool is available from
%\url{http://www.astromatic.net/software/sextractor}} and STIFF, of images
%retrieved from SDSS\footnote{The SDSS image server is available from
%\url{http://casjobs.sdss.org/vo/DR7SIAP/SIAP.asmx}}, 2MASS\footnote{The 2MASS
%image server is available from
%\url{http://irsa.ipac.caltech.edu/applications/2MASS/IM/}} and HST\footnote{The
%HST image server is available from
%\url{http://archive.stsci.edu/siap/search.php}} image servers.

%\item \textbf{PyMorph Service}:
%a software that allows to derive morphological parameters for galaxy images. Is
%possible to provide to it the output FITS files generated by Mosaic Service.

%\item \textbf{VOPlot}:
%a software tool developed in JAVA that allows to visualize astronomical data
%available in VOTable, ASCII and FITS formats.

%\item \textbf{VOMegaPlot}:
%a software tool developed in JAVA that allows to visualize astronomical data
%available in VOTable format. It looks just like VOPlot. There is a client-server
%version.

%\item \textbf{AstroStat}:
%a software tool that allows astronomers to use both and sophisticated statical
%routines on large datasets uploaded in VOTable or ASCII format.

%\item \textbf{VOCat}:
%a software tool that converts astronomical catalogs to MySQL databases. 

%\item \textbf{VOPlatform}:
%a software tool developed in JAVA that allows to place their frequently used VO
%tools and datasets with others resourcers like documents, links, among other.

%\item \textbf{VOConvert (ConVOT)}:
%a software tool that converts ASCII to VOTable files, FITS to VOTable and
%VOTable to ASCII.

%\item \textbf{Android Cosmological Calculator}:
%an Android aplication that allows to input the Hubble constant, $ \Omega_{m} $
%(matter), $ \Omega_{\lambda} $ (vacuum) and the redshift($ z $), and returns the
%current age of the Universe, the co-moving radial distance and volume and the
%angular size distance at the specified redshift, and the luminosity distance.

%\item \textbf{Android Name Resolver}:
%an Android application that allows to input the name of celestial object and
%returns information of this like RA/DEC values, redshift, proper motion,
%parallax, among other.

%\item \textbf{CSharpFITS Package}:
%a C\# .NET port of Tom McGlynn's nom.tam.fits JAVA packages\footnote{The C\#
%.NET port of Tom McGlynn's nom.tam.fits JAVA packages are availablre from
%\url{http://heasarc.gsfc.nasa.gov/docs/heasarc/fits/java/v0.9/javadoc/}}.

%\item \textbf{VOTable JAVA Streaming Writer}:
%a software that converts data streams in non-VOTable format, like ASCII or FITS,
%to the VOTable format.

%\item \textbf{C++ parser for VOTable}:
%a C++ library to access VOTable files. It has a non-streaming and streaming
%version.

%\item \textbf{Fits Manager}:
%a web-based tool for viewing, creating, adding extensions and converting FITS
%files.

%\item \textbf{HCT Data Archive System}:
%a web-based system that archives the observational data generated by the
%Himalayan Chandra Telescope (HCT), a 2 [m] aperture optical-infrared telescope
%manufactured by the EOS Technologies Inc. and remotely operated via dedicated
%satellite link.
%\end{itemize}

%%%%%%%%%%%%%%%%%%%%%%%%%%%%%%%%%%%%%%%%%%%%%%%%%%%%%%%%%%%%%%%%%%%%%%%%%%%%%%%%%
%%%%%%%%%%%%%%%%%%%%%%%%%%%%%%%Grid and Cloud%%%%%%%%%%%%%%%%%%%%%%%%%%%%%%%%%%%%
%%%%%%%%%%%%%%%%%%%%%%%%%%%%%%%%%%%%%%%%%%%%%%%%%%%%%%%%%%%%%%%%%%%%%%%%%%%%%%%%%

\subsection{Grid and Cloud}
\begin{table*}[h!t]
	\centering
	\begin{tabular}{|l|p{3cm}|p{12.5cm}|}
	\hline
	\textbf{VO} & \textbf{Project} & \textbf{Description}\\
	\hline
	VAO 	& Cross-Comparision Tool & a web tool that performs croos-comparisons between one table supplied by the user and other of an online source 
								catalog, for a user-specified match radius. This returns the all sources in the online catalog that are within the radius.\\
	\hline
	BRAVO	& BRAVO@INPE & Generate investment in information technology on Computational Infraestructure, Data Grid, Data Processing and Data Mining.\\
	\hline
	ChiVO	& Automatic Astronomical Different Scales Structures Detection and Classification within Astronomical Images & A magister thesis that intends to 
								make a software tool that finds directly astronomical objects within astronomical images through the wavelet mathematical 
								tool and a machine learning system.\\
			& Indexing of Astronomical Objects & A degree thesis that intends to design and implement an software tool that allow make an R-tree index of 
								FITS astronomical images based on their celestial coordinates. \\
	\hline
	HVO		& Photometric Redshift Estimation & a proposal that intends to execute as a web service a method developed by themselves that is capable to 
								estimate redshift from photometry increasing by two orders of magnitude the objects number of known distance. \\
			& Linking WebServices to GRID clusters & a proposal that intends, among other, to improvement the operating systems inter communication, because 
								there are simulations optimized for differents SOs and the rewritten of the codes for one different in some cases results a 
								inaccessible task.\\
			& Information Bulletin on Variable Stars & a bulletin on benhalf of the Commission 27\footnote{International Astronomical Union. (2005, November)
								. Commission 27. Variable Stars (Etoiles Variables). Retrieved from \url{http://www.konkoly.hu/IAUC27/}} and 42
								\footnote{International Astronomical Union. (2014). Comision 42: Close Binary Stars. Retrieved from: 
								\url{http://www.konkoly.hu/IAUC42/}} of the International Astronomical Union (IAU), published by the Konkoly Observatory 
								of the Hungarian Academy of Sciences. \\
	\hline
	AstroGrid& AstroRuntime & An API implemented in JAVA wich facilitates the access to the \textbf{VODesktop} services from almost any programming language
								\footnote{On the AstroGrid's official website there is a document about how to access VODesktop using Python script at 
								\url{http://www.astrogrid.org/agpython.html}}.\\
	\hline
	SVO		& VOSA & a tool that allows to analyze stellar and galactic data reading user photometry-tables, querying ``several photometrical 
								catalogs accessible through VO services'', querying ``VO-compliant theorical models (spectra)'', performing ``a 
								statistical test to determinate which model reproduced best observed data''\footnote{The VOSA tool is available from 
								\url{http://svo2.cab.inta-csic.es/theory/vosa/helpw.php?action=help2&what=intro&otype=star}}, among other. \\
			& VOSED & A service that builds Spectral Energy Distributions (SEDs) gathering information from the spectrocopic services in VO. It has two
								modes depending of the query objects number. \\
	\hline
	RVO		& INFOSEM & It aims to ``investigate, prototype and disseminate methodologies and basic techniques allowing construction of semantically 
								interoperable information systems based on the pre-existing heterogeneous information resources''\footnote{Synthesis Group. 
								(n.d.). Information System Semantic Interoperability (INFOSEM). Retrieved from 
								\url{http://synthesis.ipi.ac.ru/synthesis/projects/InfoSem/}}. \\
			& SEMIMOD & ``Modelling and Management of Semi-Structured Data for Dynamic [World Wide Web applications]''. \\
			& BIOMED & ``Methods and tools for development of subject mediators of he\-te\-ro\-ge\-neous information collections for distributed digital 
								libraries''.\\
			& REFINE & ``Modeling of compositional specifications intended for automated proof of correctness of refinement of specifications of 								requirements by pre-existing components in course of a compositional development of information systems''.\\
			& VOINFRA & ``Devolopment of principles and fundamentals of the information interoperability in the infraestructure'' of the RVO. \\
			& MULTISOURCE & ``Methods for organization of problems solving over multiple distributed heterogeneous information sources''.\\
			& RVOAG & A RVO public utility center based on AstroGrid. \\
			& ASTROMEDIA & ``Methods and tools for supporting subject mediators architecture in AstroGrid infrastructure'' for the RVO.\\
			& UNIMOD & ``Development and prototyping of experimental system for constructing the unifying information representation models for 
								interoperable integrating systems of heterogeneous information sources''.\\
			& SEMID & ``Research and development of methods and tools for semantic identification of specifications of heterogeneous information resources 
								relevant to a scientific problem and their integration in the specifications of the problem at the scientific information 
								systems''.\\
			& SubjMed & ``Investigation of methods and tools for subject mediation middleware aimed at problems solving over heterogeneous distributed 
								information resources''.\\
			& ConcMod & ``Development of methods and tools for definition of scientific subject domains conceptual models and problems solving support based 
								on mediators subject in the hybrid grid-infrastructure''.\\
			& RuleInt & ``Integration of rule-based declarative programs and knowledge databases and services for scientific problems solving over 
								heterogeneus distributed information resources''. \\
			& ASTROMEDIA Trial & ``Hybrid architecture of AstroGrid and Mediator Middlewere''. \\
			& Star Classification & ``Eclipsing-binary Stars Classification applying Ensembled Weka [algorithm] in AstroGrid''\footnote{Synthesis Group. 
								(n.d.). Information System Semantic Interoperability (INFOSEM). Retrieved from 
								\url{http://synthesis.ipi.ac.ru/synthesis/projects/InfoSem/}}.\\
	\hline
	\end{tabular}
	\caption{Grid \& Cloud}
	\label{table:gc}
\end{table*}
%\begin{itemize}
%\item \textbf{VAO}
%\item \textbf{Cross-Comparision Tool}:
%a web tool that performs croos-comparisons between one table supplied by the
%user and other of an online source catalog, for a user-specified match radius.
%This returns the all sources in the online catalog that are within the radius.

%\item \textbf{BRAVO}
%\item \textbf{BRAVO@INPE}:
%generate investment in information technology on Computational Infraestructure,
%Data Grid, Data Processing and Data Mining.

%\item \textbf{ChiVO}
%\item \textbf{Automatic Astronomical Different Scales Structures Detection and
%Classification within Astronomical Images}:
%a magister thesis that intends to make a software tool that finds directly
%astronomical objects within astronomical images through the wavelet mathematical
%tool and a machine learning system.

%\item \textbf{Indexing of Astronomical Objects}:
%a degree thesis that intends to design and implement an software tool that allow
%make an R-tree index of FITS astronomical images based on their celestial
%coordinates.

%\item \textbf{HVO}
%\item \textbf{Photometric Redshift Estimation}:
%a proposal that intends to execute as a web service a method developed by
%themselves that is capable to estimate redshift from photometry increasing by
%two orders of magnitude the objects number of known distance. 

%\item \textbf{Linking WebServices to GRID clusters}:
%a proposal that intends, among other, to improvement the operating systems inter
%communication, because there are simulations optimized for differents SOs and
%the rewritten of the codes for one different in some cases results a
%inaccessible task.

%\item \textbf{Information Bulletin on Variable Stars}:
%a bulletin on benhalf of the Commission 27\footnote{International Astronomical
%Union. (2005, November). Commission 27. Variable Stars (Etoiles Variables).
%Retrieved from \url{http://www.konkoly.hu/IAUC27/}} and
%42\footnote{International Astronomical Union. (2014). Comision 42: Close Binary
%Stars. Retrieved from: \url{http://www.konkoly.hu/IAUC42/}} of the International
%Astronomical Union (IAU), published by the Konkoly Observatory of the Hungarian
%Academy of Sciences. 

%\item \textbf{AstroGrid}
%\item \textbf{AstroRuntime}:
%an API implemented in JAVA wich facilitates the access to the \textbf{VODesktop}
%services from almost any programming language \footnote{On the AstroGrid's
%official website there is a document about how to access VODesktop using Python
%script at \url{http://www.astrogrid.org/agpython.html}}.

%\item \textbf{SVO}
%\item \textbf{VO Sed Analyzer (VOSA)}:
%a tool that allows to analyze stellar and galactic data reading user
%photometry-tables, querying ``several photometrical catalogs accessible through
%VO services'', querying ``VO-compliant theorical models (spectra)'', performing
%``a statistical test to determinate which model reproduced best observed
%data''\footnote{The VOSA tool is available from
%\url{http://svo2.cab.inta-csic.es/theory/vosa/helpw.php?action=help2&what=intro&otype=star}},
%among other. 

%\item \textbf{VOSED}:
%a service that builds Spectral Energy Distributions (SEDs) gathering information
%from the spectrocopic services in VO. It has two modes depending of the query
%objects number.

%\item \textbf{RVO}
%\item \textbf{INFOSEM}:
%it aims to ``investigate, prototype and disseminate methodologies and basic
%techniques allowing construction of semantically interoperable information
%systems based on the pre-existing heterogeneous information
%resources''\footnote{Synthesis Group. (n.d.). Information System Semantic
%Interoperability (INFOSEM). Retrieved from
%\url{http://synthesis.ipi.ac.ru/synthesis/projects/InfoSem/}}.

%\item \textbf{SEMIMOD}:
%``Modelling and Management of Semi-Structured Data for Dynamic [World Wide Web
%applications]''.

%\item \textbf{BIOMED}:
%``Methods and tools for development of subject mediators of
%he\-te\-ro\-ge\-neous information collections for distributed digital
%libraries''.
%
%\item \textbf{REFINE}:
%``Modeling of compositional specifications intended for automated proof of
%correctness of refinement of specifications of requirements by pre-existing
%components in course of a compositional development of information systems''.
%
%\item \textbf{VOINFRA}:
%``Devolopment of principles and fundamentals of the information interoperability
%in the infraestructure'' of the RVO.
%
%\item \textbf{MULTISOURCE}:
%``Methods for organization of problems solving over multiple distributed
%heterogeneous information sources''.
%
%\item \textbf{RVOAG}:
%a RVO public utility center based on AstroGrid.
%
%\item \textbf{ASTROMEDIA}:
%``Methods and tools for supporting subject mediators architecture in AstroGrid
%infrastructure'' for the RVO.
%
%\item \textbf{UNIMOD}:
%``Development and prototyping of experimental system for constructing the
%unifying information representation models for interoperable integrating systems
%of heterogeneous information sources''.
%
%\item \textbf{SEMID}:
%``Research and development of methods and tools for semantic identification of
%specifications of heterogeneous information resources relevant to a scientific
%problem and their integration in the specifications of the problem at the
%scientific information systems''.
%
%\item \textbf{SubjMed}:
%``Investigation of methods and tools for subject mediation middleware aimed at
%problems solving over heterogeneous distributed information resources''.
%
%\item \textbf{ConcMod}:
%``Development of methods and tools for definition of scientific subject domains
%conceptual models and problems solving support based on mediators subject in the
%hybrid grid-infrastructure''.
%
%\item \textbf{RuleInt}:
%``Integration of rule-based declarative programs and knowledge databases and
%services for scientific problems solving over heterogeneus distributed
%information resources''.
%
%\item \textbf{ASTROMEDIA Trial}:
%``Hybrid architecture of AstroGrid and Mediator Middlewere''.
%
%\item \textbf{Star Classification}:
%``Eclipsing-binary Stars Classification applying Ensembled Weka [algorithm] in
%AstroGrid''\footnote{Synthesis Group. (n.d.). Information System Semantic
%Interoperability (INFOSEM). Retrieved from
%\url{http://synthesis.ipi.ac.ru/synthesis/projects/InfoSem/}}.
%
%\end{itemize}

%%%%%%%%%%%%%%%%%%%%%%%%%%%%%%%%%%%%%%%%%%%%%%%%%%%%%%%%%%%%%%%%%%%%%%%%%%%%%%%%%
%%%%%%%%%%%%%%%%%%%%%%%%%%%%%%%Scientific Project%%%%%%%%%%%%%%%%%%%%%%%%%%%%%%%%
%%%%%%%%%%%%%%%%%%%%%%%%%%%%%%%%%%%%%%%%%%%%%%%%%%%%%%%%%%%%%%%%%%%%%%%%%%%%%%%%%

%\subsection{Scientific Project \& Tools}
%\begin{table*}[h!t]
%	\centering
%	\begin{tabular}{|l|p{4cm}|p{10cm}|}
%	\hline
%	VAO		& Iris: SED Analysis Tool & A downloadable application for the finding, plotting and fitting the Spectral Energies Distributions (SEDs). \\
%	\hline
%	BRAVO	& BRAVO@UFSC & Researching of the of the power spectral synthesis as a mean to estimate the physical properties of the galaxies.\\
%			& CYCLOPS & A software that models the optical emission from AM Her systems including the four Stokes parameters.\\
%	\hline
%	ChiVO	& Conceptual Design for the Search for Astronomical Patterns & A degree thesis that through the researching and analysis o search and
%							recognition methods of patterns intends to minimize the search space with minimal impact on the results. \\
%	\hline
%	ArVO	& ``Search for new interesting objects of definite types by low-dispersion template spectra'' & ``modeling of spectra [...] [for a] QSOs,
%							Seyfert galaxies, white dwarfs, [...] late-type stars (K-M, S, carbon)'' \\
%			& ``Optical identifications of new gamma, X-ray, IR and radio sources'' & using the Byurakan 2.6 [m] telescope, ``the first test resulted in 145 
%							objects found, 81 being known QSOs/Sys, and 64 new candidates (including 23 NVSS and FIRST radio sources)''.\\
%			& ``Identification of the newly found IR sources from Spitzer Space Telescope (SST)''& ``73 unidentified sources in the Bootes region have been 
%							found and clasified on the DFBS plates''\footnote{Mickaelian, A. (2006, August). Science projects with the Armenian Virtual 
%							Observatory (Arvo). Karel A.  van der Hucht (Ed.), \textit{Highlights of Astronomy} (p. 529). Vol. 14. Prague: Cambridge 
%							University Press.}.\\
%	\hline
%	ESA-VO	& Science Activities in the VO & development at the European Space Astronomy Centre (ESAC) of research projects based on VO, tutorial that teachs to use the tools made by the Science Archives Team (SAT), among other.\\
%	\hline
%	\end{tabular}
%	\caption{Scientific Project \& Tools}
%	\label{table:st}
%\end{table*}

%\begin{itemize}
%\item \textbf{VAO}
%\item \textbf{Iris: SED Analysis Tool}:
%a downloadable application for the finding, plotting and fitting the Spectral
%Energies Distributions (SEDs). 

%\item \textbf{BRAVO}
%\item \textbf{BRAVO@UFSC}:
%researching of the of the power spectral synthesis as a mean to estimate the
%physical properties of the galaxies.

%\item \textbf{CYCLOPS}:
%a software that models the optical emission from AM Her systems including the
%four Stokes parameters.

%\item \textbf{ChiVO}
%\item  \textbf{Conceptual Design for the Search for Astronomical Patterns}:
% a degree thesis that through the researching and analysis o search and
%recognition methods of patterns intends to minimize the search space with
%minimal impact on the results. 

%\item \textbf{ArVO}
%\item \textbf{``Search for new interesting objects of definite types by
%low-dispersion template spectra''}:
%``modeling of spectra [...] [for a] QSOs, Seyfert galaxies, white dwarfs, [...]
%late-type stars (K-M, S, carbon)'' 

%\item \textbf{``Optical identifications of new gamma, X-ray, IR and radio
%sources''}:
%using the Byurakan 2.6 [m] telescope, ``the first test resulted in 145 objects
%found, 81 being known QSOs/Sys, and 64 new candidates (including 23 NVSS and
%FIRST radio sources)''.

%\item \textbf{``Identification of the newly found IR sources from Spitzer Space
%Telescope (SST)''}:
%``73 unidentified sources in the Bootes region have been found and clasified on
%the DFBS plates''\footnote{Mickaelian, A. (2006, August). Science projects with
%the Armenian Virtual Observatory (Arvo). Karel A.  van der Hucht (Ed.),
%\textit{Highlights of Astronomy} (p. 529). Vol. 14. Prague: Cambridge University
%Press.}.

%\item \textbf{ESA-VO}
%\item \textbf{Science Activities in the VO}:
%development at the European Space Astronomy Centre (ESAC) of research projects
%based on VO, tutorial that teachs to use the tools made by the Science Archives
%Team (SAT), among other.

%\subsection{Non-classified}
%\begin{table*}[h!t]
%	\centering
%	\begin{tabular}{|l|l|p{12.5cm}|}
%	\hline
%	EURO-VO	& VOTECH & The first project that implemented the concept of the EURO-VO Technology Centre (EURO-VOTC) as part of the Euro-VO.\\
%			& EuroVO-ICE & as a Coordination Action supported by the European Union (UE) in the framework of the FP7 INFRA-2010-2.3.3 Research 
%					Infrastructures initiative (project 261541). It began on 1st of September, 2010 and ended on 31th of August, 2012, and succeed 
%					EuroVO-AIDA and the EuroVO-DCA.\\
%			& EuroVO-CoSADIE & As a Coordination Action supported by the EU in the framework of the FP7 INFRA-2012-3.3 Research Infraestructure 
%					initiative (project 312559). It began on 1st of September, 2012, and will end on 31th of August, 2014.\\
%	\hline
%	\end{tabular}
%	\caption{Non-classified}
%	\label{table:nc}
%\end{table*}

%\item \textbf{EURO-VO}
%\item \textbf{VOTECH}:
%the first project that implemented the concept of the EURO-VO Technology Centre
%(EURO-VOTC) as part of the Euro-VO.

%\item \textbf{Euro-VO International Cooperation Empowerment (EuroVO-ICE)}:
%as a Coordination Action supported by the European Union (UE) in the framework
%of the FP7 INFRA-2010-2.3.3 Research Infrastructures initiative (project
%261541). It began on 1st of September, 2010 and ended on 31th of August, 2012,
%and succeed EuroVO-AIDA and the EuroVO-DCA.

%\item \textbf{EuroVO-CoSADIE}:
%as a Coordination Action supported by the EU in the framework of the FP7
%INFRA-2012-3.3 Research Infraestructure initiative (project 312559). It began on
%1st of September, 2012, and will end on 31th of August, 2014.

%\end{itemize}
