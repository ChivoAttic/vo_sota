\section{Virtual Observatory Services and Tools}

The IVOA architecture only defines a general framework and
standards that their members should follow, but each VO develop
their own services and tools depending on their specific goals.
In this section, most of these initiatives are briefly
described by region to grasp the idea of the advances of
the VO.

\subsection{North America}


The \textbf{Canadian Virtual Observatory} (CVO) has largely focus on the CANFAR Virtual Storage
System\footnote{\url{http://www.canfar.phys.uvic.ca/canfar/}}, which
allows accessing very large resources for both storage and processing, 
using a cloud based framework \cite{}. 
%Virtualization and Grid Utilization within the CANFAR Project
%Gaudet, ADASS
%http://aspbooks.org/custom/publications/paper/442-0061.html
This is a very generic framework for accessing and processing 
large astronomical data sets that implements most of the
VOSpace standard of IVOA \cite{VOSPace}. CVO also has implemented
IVOA's data access services like TAP for metadata queries 
and SIA for
image access.

A central objective of any VO is to make data accessible by astronomers,
for which the \textbf{US Virtual Astronomical Observatory} (US-VAO) 
offers a web-based service called Data Discovery Tool 
that retrieve data contained in the VO~\cite{}. 
% http://arxiv.org/pdf/1206.4493v1.pdf
As a complement, a specific service for discovering time-series data
is also available, called Time Series Search Tool \cite{}.
%http://arxiv.org/pdf/1206.4035v1.pdf
In terms of cloud data processing, the NVO has a Cross-Comparison Tool 
which perform fast positional cross-matches between a large number of 
sources.
%http://www.usvao.org/documents/CrossComparisonTool/Tutorial/20120605/CrossComparisonToolTutorial_June5.pdf
It also offers a user-side application to find, plot and fit the
Spectral Energies Distributions (SEDs), called Iris~\cite{}.
%http://cxc.cfa.harvard.edu/iris/v2.0.1/publications/files/P020.pdf

\subsection{Europe}

One of the most mature VO is \textbf{Astrogrid} (United Kingdom), which implements
%http://proceedings.spiedigitallibrary.org/proceeding.aspx?articleid=874899
several IVOA-compliant services and offer very popular user applications
for accessing the VO services~\cite{}.
%http://arxiv.org/pdf/0905.2020v1.pdf
The implemented services are: \emph{Registry}
for discovering VO services, \emph{Community} to manage user accounts, 
\emph{VOSpace} for virtual file systems and processing, \emph{CEA} for running
generic asynchronous user applications, and \emph{DSA} for metadata/database
access. The very successful stand-alone applications of Astrogrid are 
the VODesktop~\cite{} for data and
%http://arxiv.org/abs/0906.1535
TOPCAT \cite{} for metadata.
%http://aspbooks.org/a/volumes/table_of_contents/?book_id=420
Between other initiatives, Astrogrid offers a middleware platform 
that offers an API for accessing VO services, and a specific 
library for Python scripting.

The \textbf{German Astrophysical Virtual Observatory} (GAVO) is other important VO in Europe that focuses not
only in observational data, but also in theoretical data \cite{}.
%http://pubman.mpdl.mpg.de/pubman/item/escidoc:1759789:1/component/escidoc:1759788/GES_paper101.pdf
The GAVO Data Center is the service that implements several IVOA standards for 
accessing to the ROSAT archive and other data, for example 
\emph{Registry}, \emph{SIAP}, \emph{SCS}, \emph{VOTable}, \emph{VOPlot}, 
\emph{TAP}, \emph{SSAP}, etc. The services for 
theoretical data includes access to the MultiDark Simulation Database \cite{}
% Search
MPA Simulations \cite{},
% search
RAVE \cite{},
% search
Millenium data \cite{},
% search
and TheoSSA \cite{}.
% search
Moreover, GAVO offers a full distribution of their server software, called
DaCHS, and maintains SPLAT (a VO-enable spectral analysis tool) and the command line TAP
client \texttt{tapsh}.

\rem{MA}{VObs.it. The webpage is not working. Commented in the code is what
Jmarroquin wrote}

%\begin{itemize}
%\item \textbf{SIAP}:
%a web services that provides the public Hubble Space Telescope/Advanced Camera
%for Surveys (HST/ACS) Great Observatories Origins Deep Survey (GOOD) data within
%the VIMOS\footnote{VIMOS is a \textbf{VI}sible imaging
%\textbf{M}ulti-\textbf{O}bject \textbf{S}pectrograph, a spectrograph for the
%European Southern Observatory Very Large Telescope array (ESO-VLT).}-VLT Deep
%Survey-Chandra Deep Field South (VVDS-CDFS).
%
%\item \textbf{SSAP}:
%a web service that allows to access the VVDS-F02-DEEP spectra.
%
%\item \textbf{CONE SEARCH}:
%a web service that allows to query in the VVDS-CDFS catalog.
%
%\item \textbf{SKYNODE}:
%a web service that allows to query in the VVDS catalogs.
%\end{itemize}


The data access services of the \textbf{Observatoires Virtuels France}
(VO-France) are mainly concentrated in the 
CDS Portal, which is a mature service that host popular web-based and
user applications such as Simbad, Aladin, Vizier, Sesame, SimPlay, X-match,
etc., following the IVOA standards. 
Just to highlight a few, Sesame is a name service that query to 
three very complete catalogs and object databases (Simbad, Vizier and Ned),
in order to resolve a name into RA/DEC coordinates. Other interesting 
service is X-match which performs a cross-matching between large amount
of CDS objects or uploaded ones using a grid-based platform.
%search
Another recent VO-France project that is not part of CDS Portal 
is GhoSST \cite{},
%search,
which is an experimental database on spectroscopy of solids, that
plans in the near future to be IVOA compliant.

Even though the \textbf{Spanish Virtual Observatory} (SVO) is a younger VO, it
provides access to numerous databases of observational data, and also manages a
theoretical database of synthetic spectra, models and even asteroseismology.
Between the offered IVOA-compliant services we found the VO SED Analyzer (VOSA)
for comparing, managing and processing user or VO photometry-tables,
%search
and the VOSED, which is a SED generator using the VO data.
%search
Also, they provide other interesting services like the Filter Profile Service
database, and the TESELA service \cite{} to acces a catalog of blank regions.
%http://arxiv.org/pdf/1107.3949v1.pdf
The SVO is moving also to data mining projects like the automated classification
of light curves, and the GAIA project that aims to produce a 3D chart of the 
Milky Way. 

\rem{MA}{UkrVo}

The \textbf{Armenian Virtual Observatory} (ArVO) is strongly based on the
Digital First/Second Byurakan Survey~\cite{}, 
%http://arxiv.org/pdf/1103.5624.pdf
%http://www.aracneeditrice.it/pdf/2421.pdf
and besides maintaining and making accessible this data to other VOs,
they are focused on generating specialized catalogs like the
blue stellar objects catalog or the late-type star catalog \cite{}.
%http://www.grid.am/pdf/Science_with_the_Armenian_Virtual_Observatory_(ArVO).pdf

The \textbf{Hungarian Virtual Observatory} (HVO) offers a web-based Spectrum
Service \cite{},
%http://www.voservices.net/spectrum/downloads/adass15_proc.pdf
that allows access, manipulation and composition of
several measured spectra of astronomical objects. It
also plan to include synthetic spectra generated by
astrophysical models that can be compared with observational
data. Other future plans are automatic photometric redshift 
estimation and performing all these computations in grid clusters. 

\subsection{Asia}

\rem{MA}{China-VO}
\rem{MA}{JVO}
\rem{MA}{RVO}
\rem{MA}{VOI}

\subsection{Oceania}
\rem{MA}{Aus-VO}


\subsection{Africa}
\rem{MA}{SA$^3$}


\subsection{South America}
\rem{MA}{BRAVO}
\rem{MA}{NOVA}
\rem{MA}{ChiVO}

\subsection{International Organizations}

The \textbf{European Virtual Observatory} (EURO-VO) offers support and funding
for the European Community VOs.  It works by executing collaboration and
development projects between the VOs with an specific topic, such as ICE
(International Cooperation Empowerment) \cite{},
%search
AIDA (Astronomical Infrastructure for Data Access) \cite{},
%search
DCA (Data Centre Alliance) \cite{}, and
%search
VOTECH (Virtual Observatory Technology) \cite{}.
Currently, the EURO-VO is executing the project CoSADIE (Collaborative and
Sustainable Astronomical Data Infrastructure for Europe), which offers 
support, documentation and guidance for astronomers and VO members.
Due its nature, the EURO-VO is a very important member for IVOA, because
it encourages the standardization and collaboration of European VOs.

The \textbf{European Space Agency Virtual Observatory} (ESA-VO) aims to be the
VO end-point for all the space-based astronomy.  It is an active member of the
EURO-VO, providing not only data access and expertise for space-based astronomy,
but generic tools that are useful for all VOs. In fact, the official resource
registry of the EURO-VO is developed and maintained by ESA-VO. 
%search
Between the tools that ESA-VO has developed, there is the \emph{DALToolKit},
that allows to publish in the VO following the DAL protocol, and \emph{VOSpec},
which is a multi-wavelength spectral analysis tool with access to atomic and
molecular databases, spectra and theoretical models registered in the VO.
%search

