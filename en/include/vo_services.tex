\section{Virtual Observatory Services and Tools}

The IVOA architecture only defines a general framework and
standards that their members should follow, but each VO develop
their own services and tools depending on their specific goals.
In this section, most of these initiatives are briefly
described by region to grasp the idea of the advances of
the VO.

\subsection{North America}

Both the CVO and the NVO are very active members of IVOA,
providing several data access, imaging and analysis services 
and tools. 

The CVO has largely focus on the CANFAR Virtual Storage
System\footnote{\url{http://www.canfar.phys.uvic.ca/canfar/}}, which
allows accessing very large resources for both storage and processing, 
using a cloud based framework \cite{}. 
%Virtualization and Grid Utilization within the CANFAR Project
%Gaudet, ADASS
%http://aspbooks.org/custom/publications/paper/442-0061.html
This is a very generic framework for accessing and processing 
large astronomical data sets that implements most of the
VOSpace standard of IVOA \cite{VOSPace}. CVO also has implemented
IVOA's data access services like TAP for metadata queries 
\cite{} and SIA for
image access \cite{}.
%IVOA standard

A central objective of any VO is to make data accessible by astronomers,
for which the NVO offers a web-based service called Data Discovery Tool 
that retrieve data contained in the VO~\cite{}. 
% http://arxiv.org/pdf/1206.4493v1.pdf
As a complement, a specific service for discovering time-series data
is also available, called Time Series Search Tool \cite{}.
%http://arxiv.org/pdf/1206.4035v1.pdf
In terms of cloud data processing, the NVO has a Cross-Comparison Tool 
which perform fast positional cross-matches between a large number of 
sources.
%http://www.usvao.org/documents/CrossComparisonTool/Tutorial/20120605/CrossComparisonToolTutorial_June5.pdf
It also offers a user-side application to find, plot and fit the
Spectral Energies Distributions (SEDs), called Iris~\cite{}.
%http://cxc.cfa.harvard.edu/iris/v2.0.1/publications/files/P020.pdf

\subsection{Europe}

ArVO is strongly based on the Digital First/Second Byurakan 
Survey~\cite{}, 
%http://arxiv.org/pdf/1103.5624.pdf
%http://www.aracneeditrice.it/pdf/2421.pdf
and besides maintaining and making accessible this data to other VOs,
they are focused on generating specialized catalogs like the
blue stellar objects catalog or the late-type star catalog \cite{}.
%http://www.grid.am/pdf/Science_with_the_Armenian_Virtual_Observatory_(ArVO).pdf

The HVO offers a web-based Spectrum Service \cite{},
%http://www.voservices.net/spectrum/downloads/adass15_proc.pdf
that allows access, manipulation and composition of
several measured spectra of astronomical objects. It
also plan to include synthetic spectra generated by
astrophysical models that can be compared with observational
data. Other future plans are automatic photometric redshift 
estimation and performing all these computations in grid clusters. 

One of the most mature VO is Astrogrid, which implements
%http://proceedings.spiedigitallibrary.org/proceeding.aspx?articleid=874899
several IVOA-compliant services and offer very popular user applications
for accessing the VO services~\cite{}.
%http://arxiv.org/pdf/0905.2020v1.pdf
The implemented services are: \emph{Registry}
for discovering VO services, \emph{Community} to manage user accounts, 
\emph{VOSpace} for virtual file systems and processing, \emph{CEA} for running
generic asynchronous user applications, and \emph{DSA} for metadata/database
access. The very successful stand-alone applications of Astrogrid are 
the VODesktop~\cite{} for data and
%http://arxiv.org/abs/0906.1535
TOPCAT \cite{} for metadata.
%http://aspbooks.org/a/volumes/table_of_contents/?book_id=420
Between other initiatives, Astrogrid offers a middleware platform 
that offers an API for accessing VO services, and a specific 
library for Python scripting.


The GAVO is other important VO in Europe that focuses not
only in observational data, but also in theoretical data \cite{}.
%http://pubman.mpdl.mpg.de/pubman/item/escidoc:1759789:1/component/escidoc:1759788/GES_paper101.pdf
The GAVO Data Center is the service that implements several IVOA standards for 
accessing to the ROSAT archive and other data, for example 
\emph{Registry}, \emph{SIAP}, \emph{SCS}, \emph{VOTable}, \emph{VOPlot}, 
\emph{TAP}, \emph{SSAP}, etc. The services for 
theoretical data includes access to the MultiDark Simulation Database \cite{}
% Search
MPA Simulations \cite{},
% search
RAVE \cite{},
%
Millenium data \cite{},
%
and TheoSSA \cite{}.
Moreover, GAVO offers a full distribution of their server software, called
DaCHS, and maintains SPLAT (a VO-enable spectral analysis tool) and the command line TAP
client \texttt{tapsh}.

\rem{MA}{SVO}
The younger SVO offers

\rem{MA}{UkrVo}

\rem{MA}{VO-France (No Info from JM, aladin and so on...)}


\subsection{Asia}

\subsection{Africa}

\subsection{Oceania}

\subsection{South America}
