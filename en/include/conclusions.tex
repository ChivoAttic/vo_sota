\section{Conclusion}
%At present the distribution of the virtual observatories in the world is not
%related with the astronomical installations, e.g., only ESO (European Souther
%Observatory) operates in three places in the north of Chile: La Silla, Paranal
%and
%Chajnantor\footnote{\url{http://www.eso.org/public/chile/about-eso/cooperation.html}},
%but there still does not exist the presence of the Virtual Observatory (VO).
%The 47\% of virtual observatories have been founded by the members of European
%Community.\\

The membership of IVOA does not
guarantee a constant contribution from its members: the alliance only 
intends to share the astronomical
knowledge between them and the community in a standardized manner. 
In fact, during this research, we have realized that several VOs have not 
updated the status of their projects, and moreover several official sources 
or data is not accessible from a web platform.
We believe that a global way of knowing the IVOA-services status would help 
assessing the VO as a whole, which is important to give the astronomer 
an integrated view about the capabilities of the VO.
 
Even though several VOs make efforts to follow the IVOA standards,
the previous section shows a clear bias to search methods in the
sky rather than searching by wavelength or other properties. 
The poor presence of SSA services in the far-infrared/millimeter
bands shows that there is a need of richer standards and data models
that allows smart-indexing of spectroscopic data cubes such as the ones ALMA is
generating. In fact, the size of ALMA data suggest that not only
smart search methods are needed, but a whole virtual infrastructure 
will be required in the near future for processing, reducing and do
science remotely in IVOA data centers rather than working locally.

As IVOA members, we believe that a proper registry service is the solution to
keep an updated VO working as a whole. However, a central IVOA status site is
needed, where the registry statistics presented by this paper could be
accessible on-line to curate the available data and to propose new projects, in
order to increase the service coverage and standardization of the final data
products. In the same line, we think that the registry service should include a
\emph{modification request} protocol, to ask the publishers for missing fields
or labels, or for correcting wrong or misleading metadata. 

Simultaneously, the effort of tracking the projects and status of the
country-VOs done by this paper must not stop here: we believe that an
updated repository of information with the activity and structure of
each VO could increase collaboration and reduce redundancy on the 
development of tools and services. However, the best way of constructing 
and maintaining such repository in a neutral basis is still an open question.

The other important non-solved issue is VO provenance and crediting. 
There are only 72 papers of 2014 (until October) in the ADS database 
that explicitly state in the text that their are using VO tools or data,
yet we believe that the real number of papers using these technologies
is much higher. This hinders the impact of the VO in science,
and could endanger the funding for developing and maintaining VO services.

%The development and implementing of a virtual observatory in Chile is urgent.
%Chile is an astronomer's
%paradise\footnote{\url{http://www.bbc.co.uk/news/world-latin-america-14205720}}.
%A platform under the IVOA's standards from there allows to facilitate the
%Chileans and global astronomical contributions, among others.\\

%Countless of tools could be developed from a Chilean virtual observatory. These
%applications would respond to the currents and future needs for any member of
%IVOA. The International Virtual Observatory Alliance has the Working Groups
%which works to development the standards that would later all members will be
%submitted.\\
