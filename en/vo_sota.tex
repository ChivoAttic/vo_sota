\documentclass[10pt]{article}
\usepackage[utf8]{inputenc}
\usepackage{amsmath}
\usepackage{epsfig}
\usepackage{enumerate}
\usepackage{float}
\usepackage{listings}
\frenchspacing
\linespread{1.2}                                          %espacio entre líneas
\setlength{\parskip}{1.5ex plus 0.2ex minus 0.2ex}        %espacio entre párrafos
\setlength{\columnsep}{0.9cm}                 %espacio entre columnas
\usepackage{indentfirst}
\usepackage{graphicx}
\usepackage{verbatim}
\usepackage{url}
\usepackage{multicol}
\usepackage{geometry}
\usepackage{fancyhdr}
\usepackage{moreverb}
\usepackage{hyperref}
\usepackage{caption}

\newcommand\R{R}
\newenvironment{keywords}{\begin{description}\item[Keywords:]}{\end{description}}

%\center{\emph{Desarrollo de una plataforma astroinformática para la administración y análisis inteligente de datos a gran escala} \\}

\title{
\center{\textbf{Presence of the Virtual Observatory in the World} \\}
\author{
        Mauricio Solar, Marcelo Mendoza, Jonathan Antognini, José Marroquín, \\
        Jorge Ibsen, Lars Nyman,
        Eduardo Vera, Diego Mardones, Guillermo Cabrera,\\
        Paola Arellano,
        Karim Pichara, Nelson Padilla,
        Ricardo Contreras, \\ Neil Nagar,
        Victor Parada.
}
\date{Valparaíso, \today}
}

\begin{document}
\maketitle

\begin{center}
    \begin{abstract}
		This document publicizes the virtual observatories projects that make
up the International Virtual Observatory Alliance (IVOA), how they are
distributed worldwide and trough some brief descriptions, the tools which they
themselves develop under standards that facilitate the sharing of astronomical
knowledge and the interoperability. For this research were reviewed the website
of the IVOA and its membership, scientific publications, articles in book and
other electronics sources. This document is required by an initiative which
intends from Chile to development of an astro-informatics platform to manage
and analyse large-scale data based on the IVOA standards, project which
involves the active participation of university students of the Federico Santa
Mar\'{i}a Technical University.
    \end{abstract}
\end{center}

\vspace{0.4cm}

\begin{center}
\begin{keywords}
    IVOA, ChiVO, VO.
\end{keywords}
\end{center}
\newpage

\section{Introduction}
	The Virtual Observatory (VO) is a international initiative which allows
the access to astronomical files and data centers to astronomers and any person
through Internet. With the standardization of the information and methods is
possible study the astronomical data without the physical requirement of the
tools and location.\\

	In June 2002, was made the International Virtual Observatory Alliance
(IVOA) to ``facilitate the international coordination and collaboration
necessary for the development and deployment of the tools, systems and
organizational structures necessary to enable the international utilization of
astronomical archives as an integrated and interoperating virtual observatory''.
Actually, the IVOA is composed of 19\footnote{On the official website in
\textbf{What is the IVOA} ``the IVOA now comprises 17 VO projects'', but in
\textbf{Member Organizations} appears 19 members listed.} projects of America,
Asia, Europa and Oceania; its members meet two times each year in
Interoperability  Workshops to have discussions face-to-face and resolve
technical questions.\\

	An initiative led by Ph. D. Mauricio Solar alongisde students of
Federico Santa Mar\'{i}a Technical University intends to develop an
astro-informatics platform to manage and analyse intelligently large-scale data
based on the IVOA standards. Due the above, this document aims:

\begin{itemize}
	\item Publicize the distribution of the virtual observatories worldwide.
	\item List the tools developed by the virtual observatories and their
status from the information provided on their official websites on Internet.
	\item Get an idea about what additional tools could be developed to
guarantee fulfilling the main objective\footnote{\textit{``Desarrollo de una
plataforma astro-inform\'{a}tica para la administraci\'{o}n y an\'{a}lisis
inteligente de datos a gran escala''} according to the name of Fondef D11$
\vert $1060 project}.
\end{itemize}

For these purposes were reviewed the official websites of the IVOA and its
members, were read sections of documents about virtual observatories like
``Virtual Observatories, Data Mining, and Astroinformatics'' of Kirk Borne,
George Mason University, among others.\\

\newpage

%\section{Distribution of IVOA Virtual Observatories Worlwide and its Projects}
\section{IVOA Virtual Observatories Worldwide}


\rem{MA}{Necessity of worldwideness}
\begin{itemize}
\item distributed astronomy facilities
\item unnecessary redundancy of observations and work
\end{itemize}

\rem{MA}{Data Deluge}

\rem{MA}{Why is difficult virtual centers worldwide}
\begin{itemize}
\item to organize the inherent diversity: different goals, working-style, budgets, language
\item why it can be done in science: public data, collaboration, etc
\end{itemize}


\subsection{The IVOA}
\rem{MA}{Check this section}

From June 2002, projects of virtual observatories have come to integrate the
International Virtual Observatory Alliance (IVOA) under the \textbf{Guidelines
for Participation\footnote{The guidelines are available in a paper in PDF and
DOC format from
\url{http://www.ivoa.net/documents/latest/IVOAParticipation.html}}}. These were
founded through national and international governmental and private programs in
collaboration with various centers of scientific studies, universities and
others. Who integrate this project, the Virtual Observatory (VO), share
knowledge between them and the community in a standardized manner. They
themselves are who develop these standards for data exchange and
interoperability. 




The table \ref{table:partners} shows the partners of IVOA to
November 2013.\\

\begin{table}%[h!t]
\centering
%\begin{tabular}{|p{7cm}|p{7cm}|}
\begin{tabular}{|l|l|}
	\hline
	\textbf{Project} & \textbf{URL} \\
	\hline
	NOVA (Argentina) & \url{http://nova.conicet.gov.ar/} \\
	\hline
	ARVO (Armenia) & \url{http://www.aras.am/Arvo/arvo.htm} \\
	\hline
	AstroGrid (United Kingdom) & \url{http://www.astrogrid.org/} \\
	\hline
	Aus-VO (Australia) & \url{http://aus-vo.org.au/} \\
	\hline
	BRAVO (Brazil) & \url{http://www.lna.br/bravo/} \\
	\hline
   CADC (Canada) &
    \url{http://www.cadc-ccda.hia-iha.nrc-cnrc.gc.ca} \\
	\hline
    ChiVO (Chile) & \url{http://www.chivo.cl/} \\
	\hline
    China-VO (China) &
    \url{http://www.china-vo.org/} \\
%	\hline
%    ESA-VO &
%    \url{http://www.sciops.esa.int/} \\
	\hline
	EURO-VO (Europe) & \url{http://www.euro-vo.org/} \\
	\hline
	GAVO (German) & \url{http://www.g-vo.org/} \\
	\hline
	HVO (Hungary) & \url{http://hvo.elte.hu/en/} \\
	\hline
	VObs.it (Italy) & \url{http://vobs.astro.it/} \\
	\hline
	JVO (Japan) & \url{http://jvo.nao.ac.jp/}\\
	\hline
	VO-France (France) & \url{http://www.france-vo.org/} \\
	\hline
	RVO (Russia) & \url{http://www.inasan.rssi.ru/eng/rvo/} \\
	\hline
	SVO (Spain) & \url{http://svo.cab.inta-csic.es/} \\
	\hline
	SA$^3$ (South Africa) & \url{http://www.sa3.ac.za/} \\
	\hline
	UkrVO (Ukrania) & \url{http://www.ukr-vo.org/} \\
	\hline
	VAO (United States) & \url{http://www.usvao.org/} \\
	\hline
	VOI (India) & \url{http://vo.iucaa.ernet.in/~voi/} \\
	\hline
\end{tabular}
\caption{IVOA's partners.}
\label{table:partners}
\end{table}

Almost half of IVOA virtual observatories are supported in Europe\footnote{The
Observatoire Virtuel France is ommited in \textbf{Europe} subsection of
\textbf{List of Virtual Observatories} section by lack of the information.} 9 of
the total; 1 belong to Africa, 1 to Australia, 2 to North America, 3 to South
America and 5 to Asia\footnote{As the mayor part of Rusia's territory is in
Asia, it will be considered like a virtual observatory of Asian continent.}. The
figure 1 shows the distribution of the IVOA's membership per continent.\\

%\begin{figure}%[h]
%\begin{center}
%	\includegraphics[scale=0.6]{img/vo_distribution.png}
%	\caption{International Virtual Observatory Alliance distribution per
%             continent.}
%\end{center}
%\end{figure}

\begin{figure}%[h]
\begin{center}
	\includegraphics[width=0.9\linewidth]{img/VO-worldwide.png}
	\caption{International Virtual Observatory Alliance presence in the world.}
\end{center}
\end{figure}


\rem{JA}{A lot of initiatives converges in one and only VO. Access worldwide to 
scientists.}


\rem{MA}{subsec: Put here the infrastructure and the organizations}
%\subsection{Infrastructure}
%\begin{itemize}
%\item \textbf{EURO-VO}
%\item \textbf{EURO-VO Data Centre Alliance (EuroVO-DCA)}:
%it is supported by the European Union (EU) in the framework of the FP6
%e-Insfraestructure Communication Network Development initiative (project
%RI031675). It began on 1st of September, 2006, and ended on 31th of December,
%2008.
%\item \textbf{EURO-VO Astronomical Infraestructure for Data Access
%              (EuroVO-AIDA)}:
%it is supported by the European Union (UE) in the framework of the FP7
%e-Infrastructure Scientific Research Repositories initiative (project
%RI2121104). It began on 1st of February, 2008, and ended on 31th of July, 2010.
%\end{itemize}

\subsection{Organizations}
\rem{MA}{Funding and Support: important actors of a VO}
\begin{itemize}
	\item \textbf{CVO}
	\item Canadian Astronomy Data Centre
	\item \textbf{VAO}
	\item National Science Foundation, NSF
	\item National Aeronautics and Space Administration, NASA 
	\item Associated Universities, Inc, AUI.
	\item Association of Universities for Reseach in Astronomy, AURA
	\item \textbf{BRAVO}
	\item Brazilian Astronomical Society, SBA
	\item National Institue for Science and Technology in Astrophysics, INCT-A
	\item \textbf{ChiVO}
	\item 5 universities, supported by ALMA, REUNA
	\item \textbf{NOVA}
	\item 8 institutions, National Universitiy of La Plata, Faculty of Astronomical 
		Sciences and Geophysics of la Plata
	\item \textbf{ArVO}
	\item Digital First Byurakan Survey, DFBS
	\item \textbf{AstroGrid}
	\item Particle Physics and Astronomy and Research Council  (PPARC)
	\item Sciency \& Technology Facilities Council (STFC)
	\item \textbf{ESA-VO}
	\item Study
	\item \textbf{EURO-VO}
	\item Continuation of Astrophysical Virtual Observatory, European Commision and six organization
	\item \textbf{GAVO}
	\item Federal Ministry of Education and Research (BMBF)
	\item \textbf{SVO}
	\item Centro de Astrobiología (INTA-CSIC)
	\item Artificial Intelligence Department of the National University of Distance Education
	\item University of Cádiz and Center of Scientific and Academic Services of Catalonia (CESCA)
	\item \textbf{VObs.it}
	\item Italian National Institute for Astrophysics
	\item Information System Units
	\item \textbf{Ukrainian}
	\item Ukrainian Astronomical Association (UAA)
	\item \textbf{SA$^3$}
	\item National Research Fundation
	\item South African Astronomical Observatory
	\item Hartebeesthoek Radio Astronomy Observatory
	\item Square Kilometer Array South Africa 
	\item \textbf{China-VO}
	\item National Astronomical Observatories
	\item Chinese Academy of Sciences
	\item \textbf{JVO}
	\item National Astronomical Observatory of Japan
	\item Fujitsu
	\item \textbf{VOI}
	\item Inter University Center for Astronomy and Astrophysics
	\item Ministry of Communication and Information Technology
	\item \textbf{Aus-VO}
	\item Linkage Infrastructure, Equipment and Facilities
\end{itemize}

\rem{MA}{subsec: Development lines of VOs and data speciality}

% If Chile became part of International Virtual Observatory Alliance, the
% distribution of IVOA's members per continent will be as shown in the figure
% 2. \\

%\begin{comment}
%\begin{figure}%[h]
%\begin{center}
%	\includegraphics[width=110mm]{img/if_chile.png}
%	\caption{International Virtual Observatory Alliance distribution per
%             continent if Chile is accepted.}
%\end{center}
%\end{figure}
%\end{comment}

%Without considering the status of the internal projects of the virtual
%observatories, the membership of Chile would contribute to the cooperation,
%development and interoperability from America in the same percent that Asia.
%Furthermore, this fact would be very significant, because a large numbers of
%astronomical centers like observatories are placed in this country.  For now, is
%intended to work with a certain quantity of data of ALMA.\\

\newpage

\section{List of IVOA Virtual Observatories}
\subsection{America}
\subsubsection{Brazilian Virtual Observatory (BRAVO)}
The BRAVO was born with the Declarations of Intentions signed on 18th of August,
2008 by six research institutes and the Brazilian Astronomical Societyi (SBA, in
its Portuguese acronym). Later, the Brazilian Virtual Observatory was founded by
the National Institute for Science and Technology in Astrophysics (INCT-A, in
its Portugese acronym).

\begin{itemize}
\item \textbf{Projects}
\begin{itemize}
\item BRAVO@IAG
\item BRAVO@INPE
\begin{itemize}
\item \textbf{Description\footnote{On the \textbf{List of IVOA Virtual
Observatories} section, a brief description of each proyect appears only if was
found some information of them in the sources.}:} generate investment in
information technology on Computational Infraestructure, Data Grid, Data
Processing and Data Mining.
\end{itemize}
\item BRAVO@LNA
\begin{itemize}
\item \textbf{Description:} making of a virtual observatory dedicated to
Southern Astrophysical Research Telescope (SOAR) data from Brazilian
astronomers.  
\end{itemize}
\item BRAVO@UFSC
\begin{itemize}
\item \textbf{Description:} researching of the of the power spectral synthesis
as a mean to estimate the physical properties of the galaxies.
\end{itemize}
\item CYCLOPS
\begin{itemize}
\item \textbf{Description:} a software that models the optical emission from AM
Her systems including the four Stokes parameters.
\end{itemize}
\end{itemize}
\end{itemize}

\subsubsection{Canadian Virtual Observatory (CVO)}
\begin{itemize}
\item \textbf{Projects}
\begin{itemize}
\item Data Sharing (VOSpace 2.0)
\begin{itemize}
\item \textbf{Description:} a service that allows users to share files and
collaborate with team members.
\end{itemize}
\item Table Access Protocol (TAP-1.0)
\begin{itemize}
\item \textbf{Description:} a service that allows the access to all the data
described by the Common Archive Observation Model (CAOM) in use at the CADC and
tables from other projects.
\end{itemize}
\item Observation Model Core Components (ObsCore-1.0)
\begin{itemize}
\item \textbf{Description:} a model that implements a standard view for
\textbf{Table Access Protocol (TAP-1.0)}.
\end{itemize}
\item Simple Image Access (SIA-1.0)
\begin{itemize}
\item \textbf{Description:} a SIA-1.0 compliant query service for easy access to
calibrated images from most our data collections.
\end{itemize}
\end{itemize}
\end{itemize}

\subsubsection{Nuevo Observatorio Virtual Argentino (NOVA)}
The NOVA was founded by eight institutions\footnote{The institutions that
founded the NOVA are the Observatorio Astron\'{o}mico de C\'{o}rdova (OAC),
Facultad de Ciencias Astron\'{o}micas y Geof\'{i}sicas de La Plata/Universidad
de Nacional de la Plata (FCAGLP/UNLP), the Instituto de Astrof\'{i}sica de La
Plata (IALP), the Instituto Argentino de Radioastronom\'{i}a (IAR), the
Instituto de Astronom\'{i}a y F\'{i}sica del Espacio (IAFE), the Instituto de
Ciencias Astron\'{o}micas, de la Tierra y del Espacio (ICAFE), the Instituto de
Astronom\'{i}a Te\'{o}rica y Experimental (IATE), and the Complejo
Astron\'{o}mico El Leoncito (CASLEO).} among which important astronomical
institutes and the National University of La Plata through the Faculty of
Astronomical Sciences and Geophysics of La Plata. It was born in January 2009.
From June 2013, the NOVA will begin its operations and intends, in addition to
provide the astronomical observations from its official website, to implement a
platform web where it can work with the data\footnote{Agencia CTyS. (2013, May
9). Instituciones astron\'{o}micas lanzan el Nuevo Observatorio Virtual
Argentino. \textit{Agencia CTyS}. Retrieved from:
\url{http://www.ctys.com.ar/index.php?idPage=20&idArticulo=2585}}.

\begin{itemize}
\item \textbf{Projects}
\begin{itemize}
\item NOVA@CASLEO
\item NOVA@IAFE
\begin{itemize}
\item \textbf{Description:} building a database for the
observations\footnote{The more than 6 terabytes of date was stored in CDs and
DVDs.} reached by the HASTA solar telescope and its applications.
\end{itemize}
\item NOVA@IALP
\item NOVA@IAR
\item NOVA@IATE
\item NOVA@ICATE
\begin{itemize}
\item \textbf{Description:} building a database for the spectroscopic
observations\footnote{Until 1987, the database was stored in photographic
plates. After that year, the information was stored in CDs and DVDs.} available
at ICATE.
\end{itemize}
\item NOVA@OAC
\item NOVA@FCAGLP
\end{itemize}
\end{itemize}

\subsubsection{US Virtual Astronomical Observatory (VAO)}
The VAO is the succesor of the NVO (National Virtual Observatory) and was
founded by the NSF and the NASA. It is in charge of the VAO, LLC, an entity
created by the Associated Universities, Inc. (AUI) and the Association of
Universities for Research in Astronomy (AURA). VAO advise the VAO Science
Council \footnote{\url{http://www.usvao.org/governance/}}. The US VO is a
co-founder of the IVOA.

\begin{itemize}
\item \textbf{Projects}
\begin{itemize}
\item Data Discovery Tool
\begin{itemize}
\item \textbf{Description:} a web tool that allows to find datasets from
astronomical collections known to the VO like the Hubble Space Telescope (HST),
the Chandra X-ray Observatory, the Spitzer Space Telescope, among other.
\end{itemize}
\item Iris: SED Analysis Tool
\begin{itemize}
\item \textbf{Description:} a downloadable application for the finding, plotting
and fitting the Spectral Energies Distributions (SEDs). 
\end{itemize}
\item Time Series Search Tool
\begin{itemize}
\item \textbf{Description:} a web tool that allows to access the time series
data sets at the Harvard Time Series Center (TSC), the NASA Exoplanet Archive
and the Catalina Real-Time Transient Survey, and analize them with the
periodogram application of the NASA Exoplanet Archive.
\end{itemize}
\item Cross-Comparision Tool
\begin{itemize}
\item \textbf{Description:} a web tool that performs croos-comparisons between
one table supplied by the user and other of an online source catalog, for a
user-specified match radius. This returns the all sources in the online catalog
that are within the radius.
\end{itemize}
\end{itemize}
\end{itemize}

\subsection{Europe}
\subsubsection{Armenian Virtual Observatory (ArVO)}
The ArVo is based on the Digital First Byukaran Survey (DFBS), a project between
Byurakan Astrophysical Observatory, Armenia; ``La Sapienza'' Universit\`{a} di
Roma, Italia; Cornell University, USA and VO-France\footnote{Mickaelian, A.,
Sargsyan, L., Gigoyan, K., Erastova, L., Sinamyan, P., Hovhannisyan, L.,
...Mykayelyan, G. (2007, December). Science with the Armenian Virtual
Observatory (ArVo).  Retrieved from
\url{http://www.grid.am/pdf/Science_with_the_Armenian_Virtual_Observatory_(ArVO).pdf}}.
Its virtual observatory was launched in February 2008\footnote{Armenian
News-NEWS.am. (2012, February 18). Armenia creates virtual observatory server.
\textit{NEWS.am}. Retrieved from \url{http://news.am/eng/news/93843.html}}.

\begin{itemize}
\item \textbf{Projects}
\begin{itemize}
\item ``Search for new interesting objects of definite types by low-dispersion
template spectra'' 
\begin{itemize}
\item \textbf{Description:} ``modeling of spectra [...] [for a] QSOs, Seyfert
galaxies, white dwarfs, [...] late-type stars (K-M, S, carbon)'' 
\end{itemize}
\end{itemize}
\begin{itemize}
\item ``Optical identifications of new gamma, X-ray, IR and radio sources''
\begin{itemize}
\item \textbf{Description:} using the Byukaran 2.6 [m] telescope, ``the first
test resulted in 145 objects found, 81 being known QSOs/Sys, and 64 new
candidates (including 23 NVSS and FIRST radio sources)''.
\end{itemize}
\end{itemize}
\begin{itemize}
\item ``Identification of the newly found IR sources from Spitzer Space
Telescope (SST)''
\begin{itemize}
\item \textbf{Description:} ``73 unidentified sources in the Bootes region have
been found and clasified on the DFBS plates''\footnote{Mickaelian, A. (2006,
August). Science projects with the Armenian Virtual Observatory (Arvo). Karel A.
van der Hucht (Ed.), \textit{Highlights of Astronomy} (p. 529). Vol. 14. Prague:
Cambridge University
Press. }.
\end{itemize}
\end{itemize}
\end{itemize}

\subsubsection{Hungarian Virtual Observatory (HVO)}
\begin{itemize}
\item \textbf{Projects}
\begin{itemize}
\item Spectrum Service for VO
\begin{itemize}
\item \textbf{Description:} a proposal that intends to add several features and
make two substantial improvements\footnote{Does not specified what several
features and the two substantial improvements.} to the web services that
contains spectra of galaxies and the other astronomical objects. 
\end{itemize}
\item Synthetic Spectrum Service
\begin{itemize}
\item \textbf{Description:} a proposal that intends to serve, as a web service,
of ready made spectra for the users.
\end{itemize}
\item Photometric Redshift Estimation
\begin{itemize}
\item \textbf{Description:} a proposal that intends to execute as a web service
a method developed by themselves that is capable to estimate redshift from
photometry increasing by two orders of magnitude the objects number of known
distance. 
\end{itemize}
\item Linking WebServices to GRID clusters
\begin{itemize}
\item \textbf{Description:} a proposal that intends, among other, to improvement
the operating systems inter communication, because there are simulations
optimized for differents SOs and the rewritten of the codes for one different in
some cases results a inaccessible task.
\end{itemize}
\item Information Bulletin on Variable Stars
\begin{itemize}
\item \textbf{Description:} a bulletin on benhalf of the Commission
27\footnote{http://www.konkoly.hu/IAUC27/} and
42\footnote{http://www.konkoly.hu/IAUC42/} of the International Astronomical
Union (IAU), published by the Konkoly Observatory of the Hungarian Academy of
Sciences. 
\end{itemize}
\item Debrecen Photoheliographic Data (DPD)
\begin{itemize}
\item \textbf{Description:} a sunspot catalogue with the heliographic positions
and the areas of sunspots. A continuation of Greenwich Photoheliographic
Results (GPR) that had been discontinued on 1976.
\end{itemize}
\end{itemize}
\end{itemize}

\subsubsection{AstroGrid}
The AstroGrid is the United Kingdom's virtual observatory. It began as a project
in 2001 and was launched in April 2008 along its working service and user
software. It has been financed by the Particle Physics and Astronomy and
Research Council (PPARC) and the Science \& Technology Facilities Council
(STFC).

\begin{itemize}
\item \textbf{Projects}
\begin{itemize}
\item Topcat
\begin{itemize}
\item \textbf{Description:} an interactive graphical viewer and editor for
tabular data for formats like FITS and VOTable.
\end{itemize}
\item VODesktop
\begin{itemize}
\item \textbf{Description:} an analysis tools wich allows limit the choice of
resources through specific data saving.
\end{itemize}
\item AstroRuntime
\begin{itemize}
\item \textbf{Description:} an API implemented in JAVA wich facilitates the
access to the \textbf{VODesktop} services from almost any programming language
\footnote{On the AstroGrid's official website there is a document about how
access VODesktop using Python script at
\url{http://www.astrogrid.org/agpython.html}}.
\end{itemize}
\end{itemize}
\end{itemize}

\subsubsection{European Space Agency Virtual Observatory (ESA-VO)}
\begin{itemize}
\item \textbf{Projects}
\end{itemize}

\subsubsection{European Virtual Observatory (EURO-VO)}
The EURO-VO is the continuation of Astrophysical Virtual Observatory (AVO). The
AVO project conducted was conceived by the European Commision and six
organizations \footnote{The six organizations that founded the AVO togheter with
the European Commision are the European Southern Observatory (ESO), the European
Space Agency (ESA), the AstroGrid, the CNRS-supported Centre de Données
Astronomiques de Strasbourg (CDS), the University Louis Pasteur; the
CNRS-supported TERAPIX astronomical data centre, the Institut d'Astrophysique;
and the Jodrell Bank Observatory, the Victoria University of Manchester.} to
research about the scientific requirements and technologies necessary to build
an Euopean virtual observatory. The ``EURO-VO aims at deploying and operational
VO in Europe''\footnote{\url{http://www.euro-vo.org/}}.\\

\begin{itemize}
\item \textbf{Projects}
\begin{itemize}
\item VOTECH
\begin{itemize}
\item \textbf{Description:} the first project that implemented the concept of
the EURO-VO Technology Centre (EURO-VOTC) as part of the Euro-VO.
\end{itemize}
\item EURO-VO Data Centre Alliance (EuroVO-DCA)
\begin{itemize}
\item \textbf{Description:} it is supported by the European Union (EU) in the
framework of the FP6 e-Insfraestructure Communication Network Development
initiative (project RI031675). It began on 1st of September, 2006, and ended on
31th of December, 2008.
\end{itemize}
\item EURO-VO Astronomical Infraestructure for Data Access (EuroVO-AIDA)
\begin{itemize}
\item \textbf{Description:} it is supported by the European Union (UE) in the
framework of the FP7 e-Infrastructure Scientific Research Repositories
initiative (project RI2121104). It began on 1st of February, 2008, and ended on
31th of July, 2010.
\end{itemize}
\item Euro-VO International Cooperation Empowerment (EuroVO-ICE)
\begin{itemize}
\item \textbf{Description:} as a Coordination Action supported by the European
Union (UE) in the framework of the FP7 INFRA-2010-2.3.3 Research Infrastructures
initiative (project 261541). It began on 1st of September, 2010 and ended on
31th of August, 2012, and succeed EuroVO-AIDA and the EuroVO-DCA.
\end{itemize}
\item EuroVO-CoSADIE
\begin{itemize}
\item \textbf{Description:} as a Coordination Action supported by the EU in the
framework of the FP7 INFRA-2012-3.3 Research Infraestructure initiative (project
312559). It began on 1st of September, 2012, and will end on 31th of August,
2014.
\end{itemize}
\end{itemize}
\end{itemize}

\subsubsection{German Astrophysical Virtual Observatory (GAVO)}
The GAVO was launched in 2003\footnote{Its first publications dates from 2003.
In 2004, H. Adorf and GAVO Team talking about ``GAVO - after one year'' at the
Astronomical Data Analysis (ADA) III Sant'Agata sui due Golfi, Italy. The birth
of this VO is inferred from the above.}. It is financed through the Federal
Ministry of Education and Research (BMBF).

\begin{itemize}
\item \textbf{Projects}
\begin{itemize}
\item GAVO Data Center
\begin{itemize}
\item \textbf{Description:} a growing collection of data and services provided
on behalf of third parties. Some of the GAVO services are also available on
\url{http://dc.zah.uni-heidelberg.de/}
\end{itemize}
\item GAVO Data Center
\begin{itemize}
\item \textbf{Description:} a collection of data and services on behalf of third
parties.
\end{itemize}
\item MPA Simulations access
\begin{itemize}
\item \textbf{Description:} a web service for querying the results of the
Millennium simulation using SQL.
\end{itemize}
\item MultiDark Database
\begin{itemize}
\item \textbf{Description:} a service wich gives access to data from MultiDark
and Bolshoi simulations using SQL queries.  It based on the Millennium Web
Application.
\end{itemize}
\item RAVE archive search
\begin{itemize}
\item \textbf{Description:} an access to a growing archive of radial velocities
for more than 400 000 stars.
\end{itemize}
\item TheoSSA
\begin{itemize}
\item \textbf{Description:} a service for providing spectral energy
distributions based on model atmosphere calculations.
\end{itemize}
\end{itemize}
\end{itemize}

\subsubsection{Observatoire Virtuel France (VO-France)}
\begin{itemize}
\item \textbf{Projects}
\end{itemize}

\subsubsection{Spanish Virtual Observatory (SVO)}
The SVO started in June 2004 and its participants are the Centro de
Astrobiolog\'{i}a (INTA-CSIC), the Artificial Intelligence Department of the
National University of Distance Education (UNED, in its Spanish acronym), the
University of C\'{a}diz and the Center of Scientific and Academic Services of
Catalonia (CESCA, in its Spanish acronym).

\begin{itemize}
\item \textbf{Projects}
\begin{itemize}
\item VO Sed Analyzer (VOSA)
\begin{itemize}
\item \textbf{Description:}  
\end{itemize}
\item VOSED
\begin{itemize}
\item \textbf{Description:} a service that builds Spectral Energy Distributions
(SEDs) gathering information from the spectrocopic services in VO. It has two
modes depending of the query objects number.
\end{itemize}
\item TESELA
\begin{itemize}
\item \textbf{Description:} a service that allows to acces the catalog of blank
regions. It is based on the application of the Delaunay triangulation of the
sky.
\end{itemize}
\end{itemize}
\end{itemize}

\subsubsection{Italian Virtual Observatory (VObs.it)}
The VObs.it besides being member of IVOA, it is a member of EURO-VO. It was
established and founded by the Italian National Institute for Astrophysics
(INAF) and it is coordinated by the Information Systems Units (SI) of this. 

\begin{itemize}
\item \textbf{Projects}
\begin{itemize}
\item SIAP
\begin{itemize}
\item \textbf{Description:} a web services that provides the public Hubble Space
Telescope/Advanced Camera for Surveys (HST/ACS) Great Observatories Origins Deep
Survey (GOOD) data within the VIMOS\footnote{VIMOS is a \textbf{VI}sible imaging
\textbf{M}ulti-\textbf{O}bject \textbf{S}pectrograph, a spectrograph for the
European Southern Observatory Very Large Telescope array (ESO-VLT).}-VLT Deep
Survey-Chandra Deep Field South (VVDS-CDFS).
\end{itemize}
\item SSAP
\begin{itemize}
\item \textbf{Description:} a web service that allows to access the
VVDS-F02-DEEP spectra.
\end{itemize}
\item CONE SEARCH
\begin{itemize}
\item \textbf{Description:} a web service that allows to query in the VVDS-CDFS
catalog. 
\end{itemize}
\item SKYNODE
\begin{itemize}
\item \textbf{Description:} a web service that allows to query in the VVDS
catalogs. 
\end{itemize}
\end{itemize}
\end{itemize}

\subsection{Asia}
\subsubsection{Chinese Virtual Observatory (China-VO)}
The China-VO was initiated in 2002 by the National Astronomical Observatories,
Chinese Academy of Sciences. It has four partners\footnote{The partners of
China-VO are the National Astronomical Observatories, Chinese Academy of
Sciences (NAOC); the TianJin University (TJU), the Central China Normal
University (CCNU), Kunming University of Science and Technology.} and more of
twelve collaborators\footnote{The China-VO's collaborators are the Computer
Network Information Center, the Purple Mountain Astro Obs, the Shangai Astro
Obs, the Yunnan Astro Obs, the Tsinghua University, the JHU, MSR, Caltech,
IUCAA, CDS, ICRAR (Australia), NAOJ (Japan), among other.}.

\begin{itemize}
\item \textbf{Projects}
\end{itemize}

\subsubsection{Japanese Virtual Observatory (JVO)}
The JVO was implemented by the National Astronomical Observatory of Japan (NAOJ)
with collaboration of Fujitsu for the development of JVO prototype systems. It
started in 2002, its VO data services are interoperated from 2004 and an
analysis system was integrated in 2005 to the JVO prototype\footnote{Shirasaki
Y., Tanaka, M., Kawamoto, S., Honda, S., Ohishi, M., Mizumoto, Y., ...Sakamoto,
M. (2006, April 8) Japanese Virtual Observatory (JVO) as an advanced
astronomical research enviroment. Retrieved from:
\url{http://arxiv.org/pdf/astro-ph/0604593v1.pdf}}.

\begin{itemize}
\item \textbf{Projects}
\begin{itemize}
\item JVO portal service
\begin{itemize}
\item \textbf{Description:} a portal site to various kind of astronomical
resources from the Subaru Telescope, Sloan Digital Sky and ALMA, among other.
\end{itemize}
\end{itemize}
\end{itemize}

\subsubsection{Russian Virtual Observatory (RVO)}
A reason for the construction of the RVO is that majority of observatories were
south of Ex-Soviet Union. After of desintegration of Ex-URSS, they were located
in the territories outside of
Russia\footnote{\url{http://www.inasan.rssi.ru/eng/rvo/project.html}}.

\begin{itemize}
\item
\textbf{Projects\footnote{\url{http://synthesis.ipi.ac.ru/synthesis/projects}}}
\begin{itemize}
\item SYNTHESIS
\begin{itemize}
\item \textbf{Description:} SYNTHESIS group's framework project.
\end{itemize}
\item INFOSEM
\begin{itemize}
\item \textbf{Description:}
\end{itemize}
\item SEMIMOD
\begin{itemize}
\item \textbf{Description:} modelling and management of semi-structured data for
dynamic World Wide Web applications.
\end{itemize}
\item BIOMED
\begin{itemize}
\item \textbf{Description:} methods and tools for development of subject
mediators of he\-te\-ro\-ge\-neous information collections for distributed
digital libraries.
\end{itemize}
\item REFINE
\begin{itemize}
\item \textbf{Description:}
\end{itemize}
\item VOINFRA
\begin{itemize}
\item \textbf{Description:} devolopment of principles and fundamentals of the
information interoperability in the infraestructure of the RVO.
\end{itemize}
\item MULTISOURCE
\begin{itemize}
\item \textbf{Description:} methods for organization of problems solving over
multiple distributed he\-te\-ro\-ge\-neous information sources.
\end{itemize}
\item RVOAG
\begin{itemize}
\item \textbf{Description:}
\end{itemize}
\item ASTROMEDIA
\begin{itemize}
\item \textbf{Description:} methods and tools for supporting subject mediators
architecture in AstroGrid\footnote{\url{http://www.astrogrid.org/}}
infrastructure for the RVO.
\end{itemize}
\item UNIMOD
\begin{itemize}
\item \textbf{Description:}
\end{itemize}
\item SEMID
\begin{itemize}
\item \textbf{Description:}
\end{itemize}
\item SubjMed
\begin{itemize}
\item \textbf{Description:}
\end{itemize}
\item ConcMod
\begin{itemize}
\item \textbf{Description:} development of methods and tools for definition of
scientific subject domains conceptual models and problems solving support based
on mediators subject in the hybrid grid-infrastructure.
\end{itemize}
\item RuleInt
\begin{itemize}
\item \textbf{Description:} integration of rule-based declarative programs and
knowledge databases and services for scientific problems solving over
heterogeneus distributed information resources.
\end{itemize}
\item ASTROMEDIA Trial
\begin{itemize}
\item \textbf{Description:} hybrid architecture of AstroGrid and Mediator
Middlewere.
\end{itemize}
\item Galaxies Search
\begin{itemize}
\item \textbf{Description:} distant galaxies search applying AstroGrid.
\end{itemize}
\item Star Classification
\begin{itemize}
\item \textbf{Description:} eclipsing-binary stars classification applying
Ensembled Weka algorithm in AstroGrid.
\end{itemize}
\end{itemize}
\end{itemize}

\subsubsection{Virtual Observatory India (VOI)}
The VOI is a collaboration between the Inter University Center for Astronomy and
Astrophysics (IUCAA) and the Persistent Systems Ltd., and is supported by the
Ministry of Communication and Information Technology, Government of India.

\begin{itemize}
\item \textbf{Projects}
\begin{itemize}
\item VOIPortal
\begin{itemize}
\item \textbf{Description:} an entry to all VOI web services. Can be browse the
data downloading or through VOIMosaic and PyMorph web applications.
\end{itemize}
\item Mosaic Service
\begin{itemize}
\item \textbf{Description:} a software that allows to make mosaic, with
SWarp\footnote{\url{http://www.astromatic.net/software/swarp}} and
SExtractor\footnote{\url{http://www.astromatic.net/software/sextractor}} and
STIFF, of images retrieved from
SDSS\footnote{\url{http://casjobs.sdss.org/vo/DR7SIAP/SIAP.asmx}},
2MASS\footnote{\url{http://irsa.ipac.caltech.edu/applications/2MASS/IM/}} and
HST\footnote{\url{http://archive.stsci.edu/siap/search.php}} image servers.
\end{itemize}
\item PyMorph Service
\begin{itemize}
\item \textbf{Description:} a software that allows to derive morphological
parameters for galaxy images. Is possible to provide to it the output FIFTS
files generated by Mosaic Service.
\end{itemize}
\item VOPlot
\begin{itemize}
\item \textbf{Description:} a software tool developed in JAVA that allows to
visualize astronomical data available in VOTable, ASCII and FITS formats.
\end{itemize}
\item VOMegaPlot
\begin{itemize}
\item \textbf{Description:} a software tool developed in JAVA that allows to
visualize astronomical data available in VOTable format. It looks just like
VOPlot. There is a client-server version.
\end{itemize}
\item AstroStat
\begin{itemize}
\item \textbf{Description:} a software tool that allows astronomers to use both
and sophisticated statical routines on large datasets uploaded in VOTable or
ASCII format.
\end{itemize}
\item VOCat
\begin{itemize}
\item \textbf{Description:} a software tool that converts astronomical catalogs
to MySQL databases. 
\end{itemize}
\item VOPlatform
\begin{itemize}
\item \textbf{Description:} a software tool developed in JAVA that allows users
to place their frequently used VO tools and datasets with others resourcers like
documents, links, among other.
\end{itemize}
\item VOConvert (ConVOT)
\begin{itemize}
\item \textbf{Description:} a software tool that converts ASCII to VOTable
files, FITS to VOTable and VOTable to ASCII.
\end{itemize}
\item Android Cosmological Calculator
\begin{itemize}
\item \textbf{Description:} an Android aplication that allows users to input the
Hubble constant, $ \Omega_{m} $ (matter), $ \Omega_{\lambda} $ (vacuum) and the
redshift($ z $), and returns the current age of the Universe, the co-moving
radial distance and volume and the angular size distance at the specified
redshift, and the luminosity distance.
\end{itemize}
\item Android Name Resolver
\begin{itemize}
\item \textbf{Description:} an Android application that allows users to input
the name of celestial object and returns information of this like RA/DEC values,
redshift, proper motion, parallax, among other.
\end{itemize}
\item CSharpFITS Package
\begin{itemize}
\item \textbf{Description:} a C\# .NET port of Tom McGlynn's nom.tam.fits JAVA
package\footnote{\url{http://heasarc.gsfc.nasa.gov/docs/heasarc/fits/java/v0.9/javadoc/}}.
\end{itemize}
\item VOTable JAVA Streaming Writer
\begin{itemize}
\item \textbf{Description:} a software that converts data streams in non-VOTable
format, like ASCII or FITS, to the VOTable format.
\end{itemize}
\item C++ parser for VOTable
\begin{itemize}
\item \textbf{Description:} a C++ library to access to VOTable files. It has a
non-streaming and streaming version.
\end{itemize}
\item Fits Manager
\begin{itemize}
\item \textbf{Description:} a web-based tool for viewing, creating, adding
extensions and converting FITS files.
\end{itemize}
\item HCT Data Archive System
\begin{itemize}
\item \textbf{Description:} a web-based system that archives the observational
data generated by the Himalayan Chandra Telescope (HCT), a 2 [m] aperture
optical-infrared telescope manufactured by the EOS Technologies Inc. and
remotely operated via dedicated satellite link.
\end{itemize}
\end{itemize}
\end{itemize}

\newpage

\section{Conclusions and Recommendations}
%At present the distribution of the virtual observatories in the world is not
%related with the astronomical installations, e.g., only ESO (European Souther
%Observatory) operates in three places in the north of Chile: La Silla, Paranal
%and
%Chajnantor\footnote{\url{http://www.eso.org/public/chile/about-eso/cooperation.html}},
%but there still does not exist the presence of the Virtual Observatory (VO).
%The 47\% of virtual observatories have been founded by the members of European
%Community.\\

The membership of IVOA does not
guarantee a constant contribution from its members: the alliance only 
intends to share the astronomical
knowledge between them and the community in a standardized manner. 
In fact, during this research, we have realized that several VOs have not 
updated the status of their projects, and moreover several official sources 
or data is not accessible from a web platform. This paper intent to be
a first step to keep an updated registry of services, tools and projects
developed by the different VOs. 

%The development and implementing of a virtual observatory in Chile is urgent.
%Chile is an astronomer's
%paradise\footnote{\url{http://www.bbc.co.uk/news/world-latin-america-14205720}}.
%A platform under the IVOA's standards from there allows to facilitate the
%Chileans and global astronomical contributions, among others.\\

%Countless of tools could be developed from a Chilean virtual observatory. These
%applications would respond to the currents and future needs for any member of
%IVOA. The International Virtual Observatory Alliance has the Working Groups
%which works to development the standards that would later all members will be
%submitted.\\

\newpage
 
\section{References}
Hanisch, R., \& Quinn, P. (n.d.) The International Virtual Observatory.
Retrieved from \url{http://www.ivoa.net/about/TheIVOA.pdf}\\

Borne, K. (n.d.) Virtual Observatories, Data Mining, and Astroinformatics. In
H.E. Bond (Ed.), \textit{Planets, Stars and Stellar Systems} (pp. 409-443).
Vol. 2. Dordrecht: Springer Science$ + $Business Media.\\

Anonymous. (2013, April 17). Proyecto Fondef: Innovador proyecto de la USM
proveer\'{a} a la Red de Telescopios ALMA de un sofisticado observatorio
virtual. Retrieved from
\url{http://www.conicyt.cl/fondef/2013/04/17/proyecto-fondef-innovador-proyecto-de-la-usm-proveera-a-la-red-de-telescopios-alma-de-un-sofisticado-observatorio-virtual/}\\

Foncea, V. (2013, April 12). Observatorios Virtuales: Desarrollo chileno de
astroinform\'{a}tica para ALMA. Retrieved from
\url{http://www.almaobservatory.org/es/anuncios-eventos/542-virtual-observatories-chilean-development-of-astronomical-computing-for-alma}\\

De Young, D. (2010, October). A Vision for the US Virtual Astronomical
Observatory. Retrieved from
\url{http://www.usvao.org/documents/VAOVisionDocumentOct2010c.pdf}\\

		Quinn, P., \& G\'{o}rsky, K.(2002). The Canadian Virtual Observatory
Project. In \textit{Toward an International Virtual Observatory}. Retrieved
from \url{http://link.springer.com/chapter/10.1007%2F10857598_4#page-1}\\

Ishihara, Y., Mizumoto, Y., Ohishi, M., Kawaray, K. (2004, September 8).
Construction of Japanese Virtual Observatory (JVO). Retrieved from
\url{http://www.fujitsu.com/downloads/MAG/vol40-2/paper05.pdf}\\
\end{document}
